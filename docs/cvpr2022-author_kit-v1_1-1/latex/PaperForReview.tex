% CVPR 2022 Paper Template
% based on the CVPR template provided by Ming-Ming Cheng (https://github.com/MCG-NKU/CVPR_Template)
% modified and extended by Stefan Roth (stefan.roth@NOSPAMtu-darmstadt.de)

\documentclass[10pt,twocolumn,letterpaper]{article}

%%%%%%%%% PAPER TYPE  - PLEASE UPDATE FOR FINAL VERSION
\usepackage[review]{cvpr}      % To produce the REVIEW version
%\usepackage{cvpr}              % To produce the CAMERA-READY version
%\usepackage[pagenumbers]{cvpr} % To force page numbers, e.g. for an arXiv version

% Include other packages here, before hyperref.
\usepackage{graphicx}
\usepackage{amsmath}
\usepackage{amssymb}
\usepackage{booktabs}


% It is strongly recommended to use hyperref, especially for the review version.
% hyperref with option pagebackref eases the reviewers' job.
% Please disable hyperref *only* if you encounter grave issues, e.g. with the
% file validation for the camera-ready version.
%
% If you comment hyperref and then uncomment it, you should delete
% ReviewTempalte.aux before re-running LaTeX.
% (Or just hit 'q' on the first LaTeX run, let it finish, and you
%  should be clear).
\usepackage[pagebackref,breaklinks,colorlinks]{hyperref}


% Support for easy cross-referencing
\usepackage[capitalize]{cleveref}
\crefname{section}{Sec.}{Secs.}
\Crefname{section}{Section}{Sections}
\Crefname{table}{Table}{Tables}
\crefname{table}{Tab.}{Tabs.}


%%%%%%%%% PAPER ID  - PLEASE UPDATE
\def\cvprPaperID{*****} % *** Enter the CVPR Paper ID here
\def\confName{CVPR}
\def\confYear{2024}


\begin{document}

%%%%%%%%% TITLE - PLEASE UPDATE
\title{Improving Pyramid Scene Parsing Network}

\author{
  Jeongho Son\\
  Department of Computer Science and Engineering, Postech\\
  {\tt\small jeonghoson@postech.ac.kr}
}

\maketitle

%%%%%%%%% ABSTRACT
\begin{abstract}
  Pyramid Scene Parsing network\cite{zhao2017pyramid} is a state-of-the-art semantic segmentation model. The key idea of this model is the Pyramid Pooling Module. The Pyramid Pooling Module calculates the global context of the image, improving the result of ResNet50\cite{he2015deep}. However, the lower layers of the Pyramid Pooling Module (2, 3, 6) layers are not utilized well. I address this by changing the structure of the Pyramid Pooling Module. This change adds minimal number of additional parameters.
\end{abstract}

%%%%%%%%% BODY TEXT
\section{Introduction}
\label{sec:intro}

One of the major issues of the ResNet50 based semantic segmentation model is the small receptive field. There are two ways to increasing the receptive field of the ResNet. The first method is to add convolutional layer. The second method is to add a pooling layer. Both methods have disadvantages. The first method increases the number of parameters. The second method reduces the resolution of the feature map. The Pyramid Scene Parsing network\cite{zhao2017pyramid}, one of the state-of-the-art semantic segmentation model, address the receptive field issue using a Pyramid Pooling Module. The Pyramid Pooling Module is a separate module only used to calculate the global context of the image. There are four layers in the Pyramid Pooling Module. I have noticed that three out of four layers are not utilized well. I address this by changing the structure of the Pyramid Pooling Module. This change adds minimal number of additional parameters.

%-------------------------------------------------------------------------
\section{Pyramid Scene Parsing Network}
\label{sec:PSPNet}

\subsection{Receptive Field}
According to the Pyramid Scene Parsing Network paper\cite{zhao2017pyramid}, there are three common issues for semantic segmentation. The issues are mismatched relationship, confusion categories, inconspicuous classes. The main cause of these issues is the small receptive field of the ResNet50\cite{he2015deep}. The model needs to consider global scene context to address the three issues stated above. The small receptive field limits the model's ability to consider global scene context leading to bad performance.

\subsection{Pyramid Pooling Module}
\begin{figure*}[h]
  \centering
  \includegraphics[width=\linewidth]{figures/PSPNet.png}
  \caption{Original PSPNet\cite{zhao2017pyramid} structure.}
  \label{fig:PSPNet}
\end{figure*}
Figure. \ref{fig:PSPNet} shows the original Pyramid Scene Parsing Network structure. The key difference for the Pyramid Scene Parsing Network is the Pyramid Pooling Module. There are multiple layers with different levels, encoding different contexts. The context encoding of the level N layer is as follows. First, the layer divides the image into N x N grid. Each grid is pooled using average pooling. Second, the context of each grid is encoded using a convolution layer with kernel size 1. In other words, there are N x N contexts, each encoding the context of the 1 / (N x N) of the image. Then, each layer is interpolated to the original size and concatenated. For example, the level 1 layer encodes the context of the whole image. The level 2 layer encodes the context of the image divided into 2 x 2 grid. There are level 1, 2, 3, 6 layers in the original Pyramid Pooling Module.

\begin{figure}
  \centering
  \includegraphics[width=\linewidth]{figures/PASCAL_VOC_horse.png}
  \caption{Example of the PASCAL VOC 2012 data.}
  \label{fig:PASCAL_VOC_horse}
\end{figure}


\begin{figure*}[h]
  \centering
  \includegraphics[width=\linewidth]{figures/PSPNet_modified.jpg}
  \caption{Modified PSPNet structure.}
  \label{fig:PSPNet_modified}
\end{figure*}

%------------------------------------------------------------------------
\section{Formatting your paper}
\label{sec:formatting}

All text must be in a two-column format.
The total allowable size of the text area is $6\frac78$ inches (17.46 cm) wide by $8\frac78$ inches (22.54 cm) high.
Columns are to be $3\frac14$ inches (8.25 cm) wide, with a $\frac{5}{16}$ inch (0.8 cm) space between them.
The main title (on the first page) should begin 1 inch (2.54 cm) from the top edge of the page.
The second and following pages should begin 1 inch (2.54 cm) from the top edge.
On all pages, the bottom margin should be $1\frac{1}{8}$ inches (2.86 cm) from the bottom edge of the page for $8.5 \times 11$-inch paper;
for A4 paper, approximately $1\frac{5}{8}$ inches (4.13 cm) from the bottom edge of the
page.

%-------------------------------------------------------------------------
\subsection{Margins and page numbering}

All printed material, including text, illustrations, and charts, must be kept
within a print area $6\frac{7}{8}$ inches (17.46 cm) wide by $8\frac{7}{8}$ inches (22.54 cm)
high.
%
Page numbers should be in the footer, centered and $\frac{3}{4}$ inches from the bottom of the page.
The review version should have page numbers, yet the final version submitted as camera ready should not show any page numbers.
The \LaTeX\ template takes care of this when used properly.



%-------------------------------------------------------------------------
\subsection{Type style and fonts}

Wherever Times is specified, Times Roman may also be used.
If neither is available on your word processor, please use the font closest in
appearance to Times to which you have access.

MAIN TITLE.
Center the title $1\frac{3}{8}$ inches (3.49 cm) from the top edge of the first page.
The title should be in Times 14-point, boldface type.
Capitalize the first letter of nouns, pronouns, verbs, adjectives, and adverbs;
do not capitalize articles, coordinate conjunctions, or prepositions (unless the title begins with such a word).
Leave two blank lines after the title.

AUTHOR NAME(s) and AFFILIATION(s) are to be centered beneath the title
and printed in Times 12-point, non-boldface type.
This information is to be followed by two blank lines.

The ABSTRACT and MAIN TEXT are to be in a two-column format.

MAIN TEXT.
Type main text in 10-point Times, single-spaced.
Do NOT use double-spacing.
All paragraphs should be indented 1 pica (approx.~$\frac{1}{6}$ inch or 0.422 cm).
Make sure your text is fully justified---that is, flush left and flush right.
Please do not place any additional blank lines between paragraphs.

Figure and table captions should be 9-point Roman type as in \cref{fig:onecol,fig:short}.
Short captions should be centred.

\noindent Callouts should be 9-point Helvetica, non-boldface type.
Initially capitalize only the first word of section titles and first-, second-, and third-order headings.

FIRST-ORDER HEADINGS.
(For example, {\large \bf 1. Introduction}) should be Times 12-point boldface, initially capitalized, flush left, with one blank line before, and one blank line after.

SECOND-ORDER HEADINGS.
(For example, { \bf 1.1. Database elements}) should be Times 11-point boldface, initially capitalized, flush left, with one blank line before, and one after.
If you require a third-order heading (we discourage it), use 10-point Times, boldface, initially capitalized, flush left, preceded by one blank line, followed by a period and your text on the same line.

%-------------------------------------------------------------------------
\subsection{Footnotes}

Please use footnotes\footnote{This is what a footnote looks like.
It often distracts the reader from the main flow of the argument.} sparingly.
Indeed, try to avoid footnotes altogether and include necessary peripheral observations in the text (within parentheses, if you prefer, as in this sentence).
If you wish to use a footnote, place it at the bottom of the column on the page on which it is referenced.
Use Times 8-point type, single-spaced.


%-------------------------------------------------------------------------
\subsection{Cross-references}

For the benefit of author(s) and readers, please use the
{\small\begin{verbatim}
  \cref{...}
\end{verbatim}}  command for cross-referencing to figures, tables, equations, or sections.
This will automatically insert the appropriate label alongside the cross-reference as in this example:
\begin{quotation}
  To see how our method outperforms previous work, please see \cref{fig:onecol} and \cref{tab:example}.
  It is also possible to refer to multiple targets as once, \eg~to \cref{fig:onecol,fig:short-a}.
  You may also return to \cref{sec:formatting} or look at \cref{eq:also-important}.
\end{quotation}
If you do not wish to abbreviate the label, for example at the beginning of the sentence, you can use the
{\small\begin{verbatim}
  \Cref{...}
\end{verbatim}}
command. Here is an example:
\begin{quotation}
  \Cref{fig:onecol} is also quite important.
\end{quotation}

%-------------------------------------------------------------------------
\subsection{References}

List and number all bibliographical references in 9-point Times, single-spaced, at the end of your paper.
When referenced in the text, enclose the citation number in square brackets, for
example~\cite{Authors14}.
Where appropriate, include page numbers and the name(s) of editors of referenced books.
When you cite multiple papers at once, please make sure that you cite them in numerical order like this \cite{Alpher02,Alpher03,Alpher05,Authors14b,Authors14}.
If you use the template as advised, this will be taken care of automatically.

\begin{table}
  \centering
  \begin{tabular}{@{}lc@{}}
    \toprule
    Method & Frobnability \\
    \midrule
    Theirs & Frumpy \\
    Yours & Frobbly \\
    Ours & Makes one's heart Frob\\
    \bottomrule
  \end{tabular}
  \caption{Results.   Ours is better.}
  \label{tab:example}
\end{table}

%-------------------------------------------------------------------------
\subsection{Illustrations, graphs, and photographs}

All graphics should be centered.
In \LaTeX, avoid using the \texttt{center} environment for this purpose, as this adds potentially unwanted whitespace.
Instead use
{\small\begin{verbatim}
  \centering
\end{verbatim}}
at the beginning of your figure.
Please ensure that any point you wish to make is resolvable in a printed copy of the paper.
Resize fonts in figures to match the font in the body text, and choose line widths that render effectively in print.
Readers (and reviewers), even of an electronic copy, may choose to print your paper in order to read it.
You cannot insist that they do otherwise, and therefore must not assume that they can zoom in to see tiny details on a graphic.

When placing figures in \LaTeX, it's almost always best to use \verb+\includegraphics+, and to specify the figure width as a multiple of the line width as in the example below
{\small\begin{verbatim}
   \usepackage{graphicx} ...
   \includegraphics[width=0.8\linewidth]
                   {myfile.pdf}
\end{verbatim}
}


%-------------------------------------------------------------------------
\subsection{Color}

Please refer to the author guidelines on the \confName\ \confYear\ web page for a discussion of the use of color in your document.

If you use color in your plots, please keep in mind that a significant subset of reviewers and readers may have a color vision deficiency; red-green blindness is the most frequent kind.
Hence avoid relying only on color as the discriminative feature in plots (such as red \vs green lines), but add a second discriminative feature to ease disambiguation.

%------------------------------------------------------------------------
\section{Final copy}

You must include your signed IEEE copyright release form when you submit your finished paper.
We MUST have this form before your paper can be published in the proceedings.

Please direct any questions to the production editor in charge of these proceedings at the IEEE Computer Society Press:
\url{https://www.computer.org/about/contact}.


%%%%%%%%% REFERENCES
{\small
\bibliographystyle{ieee_fullname}
\bibliography{egbib}
}

\end{document}
